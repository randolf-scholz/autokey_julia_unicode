% !TeX TXS-program:compile = txs:///pdflatex

%\RequirePackage{iftex}
%\ifLuaTeX
%	\RequirePackage{luatex85}
%	\RequirePackage{fontspec}
%	\AtBeginDocument{\setmainfont{Noto Color Emoji}[Renderer=Harfbuzz]}
%\fi

\NeedsTeXFormat{LaTeX2e}
\documentclass{minimal}
\usepackage{booktabs}
%\usepackage{stix}
%\usepackage{pifont}
%\usepackage[notext]{stix}
\usepackage{amsmath}
\usepackage{mathrsfs}     % math fonts
\usepackage{amsfonts}
\usepackage{pifont}
\usepackage{centernot}
\newcommand{\PrintMathFonts}{%
  \count255=0
  \loop\ifnum\count255<16
    (\the\count255:~\fontname\textfont\count255)\newline
    \advance\count255 by 1
 \repeat}


\makeatletter
\NewDocumentCommand{\alias}{m o m}{%
% Usage: \unicode@alias{<command>}[<alias>]{<fallback>}
% Checks if <alias> command is defined. if so, performs \let<command><alias>, else \let<command><fallback>
\IfValueTF{#2}{%
	\@ifundefined{#2}{%
		\PackageInfo{unicode-symbols}{Defining #1 as #3}%
		\newcommand{#1}{#3}%
	}{%
		\let#1=#2%
	}%
}{%
	\newcommand{#1}{#3}%
}%
}
\makeatother

\usepackage{etoolbox}
\makeatletter
\NewDocumentCommand{\makealias}{m O{\relax} O{\relax} O{\relax} m}{%
  % #1 is the new command (\mycommand)
  % #2 is the existing command (\aliased)
  % #3 is the fallback
\ifundef{#2}{\providecommand{#1}{FALLBACK}}{\providecommand{#1}{YES, its #2 my dear}}
%	\show#1%
%\show#2%
%\show#3%
%\show#4%
%  \ifdefined#2%
%    \let#1=#2%
%  \else\ifdefined#3\else\ifdefined#4\else
%    \providecommand{#1}{#5}
%  \fi\fi\fi%
}
\makeatother

\newcommand{\original}{This is the aliased command.}
\makealias{\aliasMISS}{This is the fallback.}
\makealias{\aliasOK}[\original]{OH NO NO NO}
\makealias{\aliasFALLBACK}[\undefinedcommand]{This is the fallback.}


\newcommand{\cmark}{\ding{51}}% ✓ (U+2713: Check Mark)
\newcommand{\xmark}{\ding{55}}% ✗ (U+2717: Ballot X)
\newcommand{\hcmark}{\ding{52}}% ✔ (U+2714: Heavy Check Mark)
\newcommand{\hxmark}{\ding{56}}% ✘ (U+2718: Heavy Ballot X)

\usepackage{amssymb,graphicx}
\begin{document}

\begin{tabular}{l}
$x^{n+1}x^{n+1}x^{n+1}x^{n+1}$
\\ $x^{n\texttt{+}1}x^{n\texttt{+}1}x^{n\texttt{+}1}x^{n\texttt{+}1}$
\\ $x^{n\mathord{\texttt{+}}1}x^{n\mathord{\texttt{+}}1}x^{n\mathord{\texttt{+}}1}x^{n\mathord{\texttt{+}}1}$
\\ $x^{n\mathbin{\texttt{+}}1}x^{n\mathbin{\texttt{+}}1}x^{n\mathbin{\texttt{+}}1}x^{n\mathbin{\texttt{+}}1}$
\end{tabular}
\aliasMISS\newline
\aliasOK\newline
\aliasFALLBACK

\begin{tabular}{ccc}\toprule
mode & text & math
\\ \midrule
   \verb|normal| & \textnormal{F}  & $\mathnormal{F}$
\\ \verb|bf|     & \textbf{F}      & $\mathbf{F}$
\\ \verb|it|     & \textit{F}      & $\mathit{F}$
\\ \verb|rm|     & \textrm{F}      & $\mathrm{F}$
\\ \verb|sf|     & \textsf{F}      & $\mathsf{F}$
\\ \verb|tt|     & \texttt{F}      & $\mathtt{F}$
\\ \verb|bb|     & --              & $\mathbb{F}$
\\ \verb|sc|     & \textsc{f}      & $--$
\\ \verb|up|     & \textup{F}      & $--$
\\ \verb|sl|     & \textsl{F}      & $--$
\\ \verb|md|     & \textmd{F}      & $--$
\\ \verb|cal|    & --              & $\mathcal{F}$
\\ \verb|src|    & --              & $\mathscr{F}$
\\ \verb|frak|   & --              & $\mathfrak{F}$
\\ \bottomrule
\end{tabular}
\PrintMathFonts
%\mathring{x}
%\acute{x}
%\bar{x}
%\breve{x}
%\check{x}
%\ddot{x}
%\dot{x}
%\grave{x}
%\hat{x}
%\tilde{x}
%\vec{x}
%\overline{x}
%\underline{x}
%\widehat{x}
%\widetilde{x}
%\overbrace{x}
%\underbrace{x}
%\overrightarrow{x}
%\overleftarrow{x}
%\mathring{x}

\end{document}
